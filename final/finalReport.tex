
%% bare_conf.tex
%% V1.3
%% 2007/01/11
%% by Michael Shell
%% See:
%% http://www.michaelshell.org/
%% for current contact information.
%%
%% This is a skeleton file demonstrating the use of IEEEtran.cls
%% (requires IEEEtran.cls version 1.7 or later) with an IEEE conference paper.
%%
%% Support sites:
%% http://www.michaelshell.org/tex/ieeetran/
%% http://www.ctan.org/tex-archive/macros/latex/contrib/IEEEtran/
%% and
%% http://www.ieee.org/

%%*************************************************************************
%% Legal Notice:
%% This code is offered as-is without any warranty either expressed or
%% implied; without even the implied warranty of MERCHANTABILITY or
%% FITNESS FOR A PARTICULAR PURPOSE!
%% User assumes all risk.
%% In no event shall IEEE or any contributor to this code be liable for
%% any damages or losses, including, but not limited to, incidental,
%% consequential, or any other damages, resulting from the use or misuse
%% of any information contained here.
%%
%% All comments are the opinions of their respective authors and are not
%% necessarily endorsed by the IEEE.
%%
%% This work is distributed under the LaTeX Project Public License (LPPL)
%% ( http://www.latex-project.org/ ) version 1.3, and may be freely used,
%% distributed and modified. A copy of the LPPL, version 1.3, is included
%% in the base LaTeX documentation of all distributions of LaTeX released
%% 2003/12/01 or later.
%% Retain all contribution notices and credits.
%% ** Modified files should be clearly indicated as such, including  **
%% ** renaming them and changing author support contact information. **
%%
%% File list of work: IEEEtran.cls, IEEEtran_HOWTO.pdf, bare_adv.tex,
%%                    bare_conf.tex, bare_jrnl.tex, bare_jrnl_compsoc.tex
%%*************************************************************************

% *** Authors should verify (and, if needed, correct) their LaTeX system  ***
% *** with the testflow diagnostic prior to trusting their LaTeX platform ***
% *** with production work. IEEE's font choices can trigger bugs that do  ***
% *** not appear when using other class files.                            ***
% The testflow support page is at:
% http://www.michaelshell.org/tex/testflow/



% Note that the a4paper option is mainly intended so that authors in
% countries using A4 can easily print to A4 and see how their papers will
% look in print - the typesetting of the document will not typically be
% affected with changes in paper size (but the bottom and side margins will).
% Use the testflow package mentioned above to verify correct handling of
% both paper sizes by the user's LaTeX system.
%
% Also note that the "draftcls" or "draftclsnofoot", not "draft", option
% should be used if it is desired that the figures are to be displayed in
% draft mode.
%
\documentclass[conference]{IEEEtran}
% Add the compsoc option for Computer Society conferences.
%
% If IEEEtran.cls has not been installed into the LaTeX system files,
% manually specify the path to it like:
% \documentclass[conference]{../sty/IEEEtran}





% Some very useful LaTeX packages include:
% (uncomment the ones you want to load)


% *** MISC UTILITY PACKAGES ***
%
%\usepackage{ifpdf}
% Heiko Oberdiek's ifpdf.sty is very useful if you need conditional
% compilation based on whether the output is pdf or dvi.
% usage:
% \ifpdf
%   % pdf code
% \else
%   % dvi code
% \fi
% The latest version of ifpdf.sty can be obtained from:
% http://www.ctan.org/tex-archive/macros/latex/contrib/oberdiek/
% Also, note that IEEEtran.cls V1.7 and later provides a builtin
% \ifCLASSINFOpdf conditional that works the same way.
% When switching from latex to pdflatex and vice-versa, the compiler may
% have to be run twice to clear warning/error messages.






% *** CITATION PACKAGES ***
%
%\usepackage{cite}
% cite.sty was written by Donald Arseneau
% V1.6 and later of IEEEtran pre-defines the format of the cite.sty package
% \cite{} output to follow that of IEEE. Loading the cite package will
% result in citation numbers being automatically sorted and properly
% "compressed/ranged". e.g., [1], [9], [2], [7], [5], [6] without using
% cite.sty will become [1], [2], [5]--[7], [9] using cite.sty. cite.sty's
% \cite will automatically add leading space, if needed. Use cite.sty's
% noadjust option (cite.sty V3.8 and later) if you want to turn this off.
% cite.sty is already installed on most LaTeX systems. Be sure and use
% version 4.0 (2003-05-27) and later if using hyperref.sty. cite.sty does
% not currently provide for hyperlinked citations.
% The latest version can be obtained at:
% http://www.ctan.org/tex-archive/macros/latex/contrib/cite/
% The documentation is contained in the cite.sty file itself.






% *** GRAPHICS RELATED PACKAGES ***
%
\ifCLASSINFOpdf
  % \usepackage[pdftex]{graphicx}
  % declare the path(s) where your graphic files are
  % \graphicspath{{../pdf/}{../jpeg/}}
  % and their extensions so you won't have to specify these with
  % every instance of \includegraphics
  % \DeclareGraphicsExtensions{.pdf,.jpeg,.png}
\else
  % or other class option (dvipsone, dvipdf, if not using dvips). graphicx
  % will default to the driver specified in the system graphics.cfg if no
  % driver is specified.
  % \usepackage[dvips]{graphicx}
  % declare the path(s) where your graphic files are
  % \graphicspath{{../eps/}}
  % and their extensions so you won't have to specify these with
  % every instance of \includegraphics
  % \DeclareGraphicsExtensions{.eps}
\fi
% graphicx was written by David Carlisle and Sebastian Rahtz. It is
% required if you want graphics, photos, etc. graphicx.sty is already
% installed on most LaTeX systems. The latest version and documentation can
% be obtained at:
% http://www.ctan.org/tex-archive/macros/latex/required/graphics/
% Another good source of documentation is "Using Imported Graphics in
% LaTeX2e" by Keith Reckdahl which can be found as epslatex.ps or
% epslatex.pdf at: http://www.ctan.org/tex-archive/info/
%
% latex, and pdflatex in dvi mode, support graphics in encapsulated
% postscript (.eps) format. pdflatex in pdf mode supports graphics
% in .pdf, .jpeg, .png and .mps (metapost) formats. Users should ensure
% that all non-photo figures use a vector format (.eps, .pdf, .mps) and
% not a bitmapped formats (.jpeg, .png). IEEE frowns on bitmapped formats
% which can result in "jaggedy"/blurry rendering of lines and letters as
% well as large increases in file sizes.
%
% You can find documentation about the pdfTeX application at:
% http://www.tug.org/applications/pdftex





% *** MATH PACKAGES ***
%
%\usepackage[cmex10]{amsmath}
% A popular package from the American Mathematical Society that provides
% many useful and powerful commands for dealing with mathematics. If using
% it, be sure to load this package with the cmex10 option to ensure that
% only type 1 fonts will utilized at all point sizes. Without this option,
% it is possible that some math symbols, particularly those within
% footnotes, will be rendered in bitmap form which will result in a
% document that can not be IEEE Xplore compliant!
%
% Also, note that the amsmath package sets \interdisplaylinepenalty to 10000
% thus preventing page breaks from occurring within multiline equations. Use:
%\interdisplaylinepenalty=2500
% after loading amsmath to restore such page breaks as IEEEtran.cls normally
% does. amsmath.sty is already installed on most LaTeX systems. The latest
% version and documentation can be obtained at:
% http://www.ctan.org/tex-archive/macros/latex/required/amslatex/math/





% *** SPECIALIZED LIST PACKAGES ***
%
%\usepackage{algorithmic}
% algorithmic.sty was written by Peter Williams and Rogerio Brito.
% This package provides an algorithmic environment fo describing algorithms.
% You can use the algorithmic environment in-text or within a figure
% environment to provide for a floating algorithm. Do NOT use the algorithm
% floating environment provided by algorithm.sty (by the same authors) or
% algorithm2e.sty (by Christophe Fiorio) as IEEE does not use dedicated
% algorithm float types and packages that provide these will not provide
% correct IEEE style captions. The latest version and documentation of
% algorithmic.sty can be obtained at:
% http://www.ctan.org/tex-archive/macros/latex/contrib/algorithms/
% There is also a support site at:
% http://algorithms.berlios.de/index.html
% Also of interest may be the (relatively newer and more customizable)
% algorithmicx.sty package by Szasz Janos:
% http://www.ctan.org/tex-archive/macros/latex/contrib/algorithmicx/




% *** ALIGNMENT PACKAGES ***
%
%\usepackage{array}
% Frank Mittelbach's and David Carlisle's array.sty patches and improves
% the standard LaTeX2e array and tabular environments to provide better
% appearance and additional user controls. As the default LaTeX2e table
% generation code is lacking to the point of almost being broken with
% respect to the quality of the end results, all users are strongly
% advised to use an enhanced (at the very least that provided by array.sty)
% set of table tools. array.sty is already installed on most systems. The
% latest version and documentation can be obtained at:
% http://www.ctan.org/tex-archive/macros/latex/required/tools/


%\usepackage{mdwmath}
%\usepackage{mdwtab}
% Also highly recommended is Mark Wooding's extremely powerful MDW tools,
% especially mdwmath.sty and mdwtab.sty which are used to format equations
% and tables, respectively. The MDWtools set is already installed on most
% LaTeX systems. The lastest version and documentation is available at:
% http://www.ctan.org/tex-archive/macros/latex/contrib/mdwtools/


% IEEEtran contains the IEEEeqnarray family of commands that can be used to
% generate multiline equations as well as matrices, tables, etc., of high
% quality.


%\usepackage{eqparbox}
% Also of notable interest is Scott Pakin's eqparbox package for creating
% (automatically sized) equal width boxes - aka "natural width parboxes".
% Available at:
% http://www.ctan.org/tex-archive/macros/latex/contrib/eqparbox/





% *** SUBFIGURE PACKAGES ***
%\usepackage[tight,footnotesize]{subfigure}
% subfigure.sty was written by Steven Douglas Cochran. This package makes it
% easy to put subfigures in your figures. e.g., "Figure 1a and 1b". For IEEE
% work, it is a good idea to load it with the tight package option to reduce
% the amount of white space around the subfigures. subfigure.sty is already
% installed on most LaTeX systems. The latest version and documentation can
% be obtained at:
% http://www.ctan.org/tex-archive/obsolete/macros/latex/contrib/subfigure/
% subfigure.sty has been superceeded by subfig.sty.



%\usepackage[caption=false]{caption}
%\usepackage[font=footnotesize]{subfig}
% subfig.sty, also written by Steven Douglas Cochran, is the modern
% replacement for subfigure.sty. However, subfig.sty requires and
% automatically loads Axel Sommerfeldt's caption.sty which will override
% IEEEtran.cls handling of captions and this will result in nonIEEE style
% figure/table captions. To prevent this problem, be sure and preload
% caption.sty with its "caption=false" package option. This is will preserve
% IEEEtran.cls handing of captions. Version 1.3 (2005/06/28) and later
% (recommended due to many improvements over 1.2) of subfig.sty supports
% the caption=false option directly:
%\usepackage[caption=false,font=footnotesize]{subfig}
%
% The latest version and documentation can be obtained at:
% http://www.ctan.org/tex-archive/macros/latex/contrib/subfig/
% The latest version and documentation of caption.sty can be obtained at:
% http://www.ctan.org/tex-archive/macros/latex/contrib/caption/




% *** FLOAT PACKAGES ***
%
%\usepackage{fixltx2e}
% fixltx2e, the successor to the earlier fix2col.sty, was written by
% Frank Mittelbach and David Carlisle. This package corrects a few problems
% in the LaTeX2e kernel, the most notable of which is that in current
% LaTeX2e releases, the ordering of single and double column floats is not
% guaranteed to be preserved. Thus, an unpatched LaTeX2e can allow a
% single column figure to be placed prior to an earlier double column
% figure. The latest version and documentation can be found at:
% http://www.ctan.org/tex-archive/macros/latex/base/



%\usepackage{stfloats}
% stfloats.sty was written by Sigitas Tolusis. This package gives LaTeX2e
% the ability to do double column floats at the bottom of the page as well
% as the top. (e.g., "\begin{figure*}[!b]" is not normally possible in
% LaTeX2e). It also provides a command:
%\fnbelowfloat
% to enable the placement of footnotes below bottom floats (the standard
% LaTeX2e kernel puts them above bottom floats). This is an invasive package
% which rewrites many portions of the LaTeX2e float routines. It may not work
% with other packages that modify the LaTeX2e float routines. The latest
% version and documentation can be obtained at:
% http://www.ctan.org/tex-archive/macros/latex/contrib/sttools/
% Documentation is contained in the stfloats.sty comments as well as in the
% presfull.pdf file. Do not use the stfloats baselinefloat ability as IEEE
% does not allow \baselineskip to stretch. Authors submitting work to the
% IEEE should note that IEEE rarely uses double column equations and
% that authors should try to avoid such use. Do not be tempted to use the
% cuted.sty or midfloat.sty packages (also by Sigitas Tolusis) as IEEE does
% not format its papers in such ways.





% *** PDF, URL AND HYPERLINK PACKAGES ***
%
%\usepackage{url}
% url.sty was written by Donald Arseneau. It provides better support for
% handling and breaking URLs. url.sty is already installed on most LaTeX
% systems. The latest version can be obtained at:
% http://www.ctan.org/tex-archive/macros/latex/contrib/misc/
% Read the url.sty source comments for usage information. Basically,
% \url{my_url_here}.





% *** Do not adjust lengths that control margins, column widths, etc. ***
% *** Do not use packages that alter fonts (such as pslatex).         ***
% There should be no need to do such things with IEEEtran.cls V1.6 and later.
% (Unless specifically asked to do so by the journal or conference you plan
% to submit to, of course. )


\usepackage{graphicx}
\usepackage{float}
\usepackage{color}
\usepackage{listings}

% correct bad hyphenation here
\hyphenation{op-tical net-works semi-conduc-tor}

\begin{document}
%
% paper title
% can use linebreaks \\ within to get better formatting as desired
\title{Comparing Application Semantics with S$^2$E \\ {\large Determining whether busybox and coreutils are actually the same}}
% or \title{Symbolically Executing Real Applications \\ {\large Trying to make S$^2$E work with real software}}

% author names and affiliations
% use a multiple column layout for up to three different
% affiliations
\author{\IEEEauthorblockN{Mariana D'Angelo}
\IEEEauthorblockA{Department of Electrical and Computer Engineering\\
University of Toronto\\
Toronto, Canada\\
mariana.dangelo@utoronto.ca}
\and
\IEEEauthorblockN{Dhaval Miyani}
\IEEEauthorblockA{Department of Electrical and Computer Engineering\\
University of Toronto\\
Toronto, Canada\\
dhaval.miyani@utoronto.ca}}

% use for special paper notices
%\IEEEspecialpapernotice{(Invited Paper)}

% make the title area
\maketitle

% For peer review papers, you can put extra information on the cover
% page as needed:
% \ifCLASSOPTIONpeerreview
% \begin{center} \bfseries EDICS Category: 3-BBND \end{center}
% \fi
%
% For peerreview papers, this IEEEtran command inserts a page break and
% creates the second title. It will be ignored for other modes.
\IEEEpeerreviewmaketitle

\section{Introduction}

While the key idea behind symbolic execution arose almost forty
years ago \cite{select}, it has only recently become practical due to
advances in constraint satisfiability \cite{satisfiability} and scalable approaches
with mixed symbolic and concrete execution \cite{systems-crash, dart}.
A 2011 study has shown that symbolic execution tools are beginning to
be used increasingly in industrial practice at corporations such as Microsoft,
NASA, IBM, and Fujitsu \cite{symb-ex-practice}. This proliferation shows the
importance of new symbolic execution tools and the impact they will have
on the industry as they improve.

S$^2$E is a symbolic execution platform that can operate directly on application
binaries for analyzing the properties and behaviour of software systems. It
analyzes programs in-vivo (the whole environment) within a real software stack
(user program, libraries, kernel, drivers, etc.) rather than using abstract models
of these layers. It is commonly used for performance profiling, reverse engineering
of proprietary software, and finding bugs in user-mode and kernel-mode
binaries \cite{s2e}. S$^2$E is a complex system due to the rich features it has
(particularly the fact that it runs inside of a VMM). This complexity, however, comes
with the price of being difficult to start using on existing software.

The goal of this experiment is to understand how a complex system such
as S$^2$E can be applied to existing software and document the process.
In particular, we are interested in seeing whether program inputs and code
need to be sanitized in some way to become amenable to dynamic analysis
via symbolic execution. Taking inspiration from KLEE \cite{klee}, we will be
comparing the semantics of two implementations of (supposedly) the exact
same suite of programs: the GNU Coreutils utility suite and the Busybox
embedded system suite.

Comparing the semantics of two implementations can have many real world
applications. In the field of computer security, for example, the ability to compare
the semantics of a software compiled from source with a pre-compiled binary
could reveal malicious or accidental backdoors. In particular, the fact that S$^2$E
runs in a VMM would allow for it to be used to test kernel level programs such as
file systems (or compare read and write operations different file systems).

We have encountered several challenges in using the S$^2$E platform; one
in particular forced us to change the initial goal of our experiment (evaluating
two consistency models from S$^2$E) due to the fact that not all of the six
consistency models described in \cite{s2e} are actually implemented in the
platform. Our biggest challenge with this project was trying to modify the programs
of interest in such a way that they could be directly comparable. The different
approaches we tried will be thoroughly discussed in the Experiments section of this
paper along with our own explanations about why they failed or succeeded and
what impact these choices had on the program analysis as a whole.

%\subsection{Subsection Heading Here}
%Subsection text here.


%\subsubsection{Subsubsection Heading Here}
%Subsubsection text here.


% An example of a floating figure using the graphicx package.
% Note that \label must occur AFTER (or within) \caption.
% For figures, \caption should occur after the \includegraphics.
% Note that IEEEtran v1.7 and later has special internal code that
% is designed to preserve the operation of \label within \caption
% even when the captionsoff option is in effect. However, because
% of issues like this, it may be the safest practice to put all your
% \label just after \caption rather than within \caption{}.
%
% Reminder: the "draftcls" or "draftclsnofoot", not "draft", class
% option should be used if it is desired that the figures are to be
% displayed while in draft mode.
%
%\begin{figure}[!t]
%\centering
%\includegraphics[width=2.5in]{myfigure}
% where an .eps filename suffix will be assumed under latex,
% and a .pdf suffix will be assumed for pdflatex; or what has been declared
% via \DeclareGraphicsExtensions.
%\caption{Simulation Results}
%\label{fig_sim}
%\end{figure}

% Note that IEEE typically puts floats only at the top, even when this
% results in a large percentage of a column being occupied by floats.


% An example of a double column floating figure using two subfigures.
% (The subfig.sty package must be loaded for this to work.)
% The subfigure \label commands are set within each subfloat command, the
% \label for the overall figure must come after \caption.
% \hfil must be used as a separator to get equal spacing.
% The subfigure.sty package works much the same way, except \subfigure is
% used instead of \subfloat.
%
%\begin{figure*}[!t]
%\centerline{\subfloat[Case I]\includegraphics[width=2.5in]{subfigcase1}%
%\label{fig_first_case}}
%\hfil
%\subfloat[Case II]{\includegraphics[width=2.5in]{subfigcase2}%
%\label{fig_second_case}}}
%\caption{Simulation results}
%\label{fig_sim}
%\end{figure*}
%
% Note that often IEEE papers with subfigures do not employ subfigure
% captions (using the optional argument to \subfloat), but instead will
% reference/describe all of them (a), (b), etc., within the main caption.


% An example of a floating table. Note that, for IEEE style tables, the
% \caption command should come BEFORE the table. Table text will default to
% \footnotesize as IEEE normally uses this smaller font for tables.
% The \label must come after \caption as always.
%
%\begin{table}[!t]
%% increase table row spacing, adjust to taste
%\renewcommand{\arraystretch}{1.3}
% if using array.sty, it might be a good idea to tweak the value of
% \extrarowheight as needed to properly center the text within the cells
%\caption{An Example of a Table}
%\label{table_example}
%\centering
%% Some packages, such as MDW tools, offer better commands for making tables
%% than the plain LaTeX2e tabular which is used here.
%\begin{tabular}{|c||c|}
%\hline
%One & Two\\
%\hline
%Three & Four\\
%\hline
%\end{tabular}
%\end{table}


% Note that IEEE does not put floats in the very first column - or typically
% anywhere on the first page for that matter. Also, in-text middle ("here")
% positioning is not used. Most IEEE journals/conferences use top floats
% exclusively. Note that, LaTeX2e, unlike IEEE journals/conferences, places
% footnotes above bottom floats. This can be corrected via the \fnbelowfloat
% command of the stfloats package.


\section{Background}

There is much work in symbolic execution for testing \cite{klee, RWset, vul-signature, system-crash, exe, bad-input, end-to-end, dyn-test, compositional-test, dart, sage, cute}. A typical approach for testing is fuzzing \cite{sage}, which involves generating random inputs for a program in an attempt to exercise all paths in a program (and hopefully hit any buggy code). This is known as concrete execution which involves running the program with deterministic values. Fuzz testing has several limitations, in particular coverage of paths in a program. For example, Listing \ref{code-fuzzing} shows a if statement which is only taken when x is 10. There are $2^{32}$ possible values which x could take when an input is randomly generated and as such the probability of actually taking this path is $2^{-32}$. This example demonstrates that the likelihood of some paths being taken when fuzzing is extremely low, which is the reason for the poor test coverage of concrete execution.

%  --------------------- CODE SYNTAX HIGHLIGHTING --------------------- %

\definecolor{mygreen}{rgb}{0,0.6,0}
\definecolor{myblue}{rgb}{0,0,0.6}
\definecolor{mymauve}{rgb}{0.58,0,0.82}
\definecolor{myred}{rgb}{0.6,0,0}
\definecolor{gray}{rgb}{0.5,0.5,0.5}
\definecolor{mygray}{gray}{0.25}

\lstdefinestyle{C}{
	language=C,
	xleftmargin=10pt,
	xrightmargin=0pt,
	basicstyle=\footnotesize\bfseries\ttfamily,
	showspaces=false,
	showtabs=false,
	breaklines=true,
	showstringspaces=false,
	breakatwhitespace=true,
	commentstyle=\color{mygreen},
	keywordstyle=\color{myblue},
	stringstyle=\color{myred},
	numbers=left,
	numbersep=5pt,
	numberstyle=\footnotesize\color{mygray},
	tabsize=4,
	frame=TB,
	framesep=5pt,
	escapeinside={\%*}{*}
}

\begin{lstlisting}[style=C, label=code-fuzzing, abovecaptionskip=2ex, captionpos=b, caption={Code example where fuzzing generally has poor coverage}]
if (x == 10) {
	// path 1
} else {
	// path 2
}
\end{lstlisting}

 Symbolic execution, on the other hand, runs the application with symbolic inputs \cite{klee} which are initially unconstrained by design. The program executes with these symbolic values and replaces concrete operations with ones that can manipulate symbolic values. The symbolic execution engine ``follows" both paths whenever it encounters a branch on a symbolic value and adds the path condition to a set of constraints known as the ``path constraint". When a bug is encountered in the code a test case (i.e., concrete inputs) can be generated from the path constraint. There are two main limitations of symbolic execution: (1) ``the path explosion problem" \cite{s2e}, which is due to the exponential growth of paths in a program caused by conditional branches, and (2) ``the environment problem" \cite{klee}, which is due to interactions with the surrounding environment (e.g, operating system, network, etc.).

 In order to deal with the environment problem, one can use a mixture of symbolic and concrete execution (also known as concolic execution \cite{cute}). Concolic execution gathers constraints using symbolic execution, but then generates concrete values using a constraint solver in order to address the environment problem. Once these concrete values are generated the program can interact with its environment in a normal fashion without the overheads and problems associated with symbolically executing the environment.

S$^2$E is a symbolic execution engine which can perform concrete execution, symbolic execution, and concolic execution depending on the consistency model chosen. There are six different consistency models in S$^2$E, described below. Figure \ref{s2e-cm} provides an overview of the models showing how they transition from highly strict to highly relaxed models with some details about what consistency requirements are relaxed for each model. A model is consistent if there exist a globally feasible path through the system for every path explored in the unit. A unit is the block one wishes to analyze and the environment is the rest of the system.

\begin{itemize}
\item \textit{Strictly Consistent Concrete Execution (SC-CE):}
No symbolic execution in unit or environment.
\item \textit{Strictly Consistent Unit-level Execution (SC-UE):}
Unit is symbolically executed while the
environment is executed concretely.
\item \textit{Strictly Consistent System-level Execution
(SC-SE):} Unit and environment are executed
symbolically -€“ this is the only model that
executes the environment symbolically.
\item \textit{Local Consistency (LC):} Similar to SC-UE,
but it adheres to constraints that the
environment/unit API contracts impose on
return values.
\item \textit{Relaxed Consistency Overapproximate
Consistency (RC-OC):} Similar to LC, but it
ignores the constraints from the environment/unit API contracts.
\item \textit{Relaxed Consistency CFG Consistency (RC-CC):}
Similar to SC-UE, but like static analysis it can explore any
path in the unit's inter-procedural control flow graph (even
infeasible ones).
\end{itemize}

There are several differences between S$^2$E and other similar symbolic execution engines. For instance, KLEE uses file system models \cite{klee} to avoid symbolically executing the actual filesystem. S$^2$E does not take this approach as writing models is a labour intensive and error-prone process. Cute \cite{cute} executes the environment concretely (i.e., without modelling) with a consistency model similar to S$^2$E's SC-SE, but it is limited to code-based selection and one consistency model. S$^2$E does not use compositional symbolic execution \cite{compositional-test}, a performance optimization which saves results for parts of the program (e.g., a function) and reuses them when that part is called in a different context. With concolic execution everything runs concretely for full systems, however, when the execution crosses program boundaries it may result in lost paths \cite{s2e}. Due to this, KLEE and CUTE cannot track path conditions in the environment and hence are unable to re-execute calls to enable overconstrained but feasible paths (e.g., malloc does not execute deterministically). DART \cite{dart}, CUTE \cite{cute} and EXE \cite{exe} use mix-mode execution (concretely executing some parts and symbolically executing others) to increase efficiency, however, they do not use automatic symbolic-concrete bidirectional data conversion (i.e., automatic conversion between concrete and symbolic values at program boundaries such as the unit and the environment) unlike S$^2$E, which is key to S$^2$E's scalability and low programmer effort.

Static analysis is performed without executing the program in question and has known limitations such as a high false positive and negative rate (due to infeasible paths) \cite{sage} and a lack of runtime environment analysis (as only the application source code is parsed). To reduce the number of false positives tools such as Saturn \cite{saturn} and bddbddb \cite{bddbddb} use a path-sensitive analysis engine. Saturn aims to detect logic programming language bugs by summarizing functions. bddbddb finds buggy patterns in a database where programs are stored as relations. As can be inferred, these tools are language specific and require different implementations for different programming languages. Additionally, using these tools requires learning a new programming language. As a dynamic analysis framework, S$^2$E addresses the limitations of static analysis tools. For example, they directly operate on and analyze binaries whereas static analysis would require disassembly and decompilation. This could involve converting an x86 binary to the LLVM format and running it through an engine like KLEE. Disassembly and decompilation \cite{disassembly} are classically undecidable problems (e.g., disambiguating code from data) \cite{s2e}.

\begin{figure}[!t]
\centering
\includegraphics[width=3in]{s2e-consistency-models.png}
\caption{S$^2$E Consistency Models \cite{s2e}}
\label{s2e-cm}
\end{figure}

\section{Approach}

Details of the research methodology or approach taken in the project.

In theory:
\begin{itemize}
  \item Run each program with a symbolic input
  \item Compare the output of the two programs with s2e\_assert (which gives more information than a regular assert)
  \item Analyze any differences to identify semantic differences between implementations (e.g., corner cases where they each handle an error differently)
\end{itemize}

\section{Status}

Current status of implementation.

\begin{itemize}
	\item It isn't actually working since our lives are awful.
	\item The world is a terrible place.
\end{itemize}

\section{Experiments}

Results of experiments that were performed.

CODE SNIPPETS MIGHT MAKE THIS LONGER - ONE PER APPROACH TO SHOW WHAT WE DID.\\
Also explanations of why things didn't work can be long and stuff.

\begin{itemize}
  \item Tried running the analysis directly on a binary (no tester program invoking the program), but could not use this to compare two programs since we weren't certain the inputs would be the same (although is this really true with symbolic execution? I HAVE NO IDEA)
  \item Tried to run the echo binaries directly using popen and fread
  \item Tried to make the echo implementations into shared libraries so they could be called directly from our test program (shared library vs. static library stuff)
  \item Once we had these applications we tried simply redirecting stdout into a buffer to read their outputs (failed for whatever reason)
  \item After all of these attempts we decided to look into how KLEE ended up doing this comparison of coreutils and busybox. We discovered that KLEE was symbolically modifying stdout (klee option {\tt -sym-stdout}). Since S2E is built upon KLEE we tried to add this flag into the klee arguments portion of the config.lua file for S2E (include listing with config file we used for our testing? or is it not relevant because testcasegeneration didn't even seem to work...), however this option did not exist, most likely due to the fact that s2e is meant to be in vivo and use a real software stack (i.e., not model any of those layers).
  \item Final approach: modifying coreutils echo and busybox echo to return a string instead of printing to stdout [WORKED]
\end{itemize}

\section{Results}

\begin{itemize}
  \item modified echo to return strings for both implementations
  \item got 4 assertions that failed between the two implementations
  \item manually inspected the programs with gdb given the inputs provided by s2e (e.g., 0x2c02) to step through the execution and see where execution differed
  \item discovered a bug in our own modifications to coreutils echo where we were adding a space instead of newlines
  \item each time we run s2e we get a different number of states even though they should hypothetically always go through the same number of states
  \item one example of a difference between the implementations of echo is that both echo's print a space between strings found in argv, however the condition preceding the addition of this space is different between the implementations (for coreutils) if there is at least one string to print (check argc > 0), a space is always added after the word whereas busybox only prints a space after a word if there is another string to print (breaks if *argv++ not NULL).
  \item two input patterns that failed the asserts (1) ended with 0x00 (i.e., NULL character), and (2) ended with extended ascii values (i.e., expressed as a signed integer -0x45 which is a symbol)
\end{itemize}

\section{Conclusion}

Did your results meet expectations?

\begin{itemize}
	\item Nothing worked, the world is a horrible place.
  \item As a sort of lessons learned: lack of documentation and an incomplete understanding of how S2E symbolically executes code led to a lot of mistakes.
  \item Also the implications of symbolic execution - do we really need to use the same symbolic input on each? If it's not concolic, does it really matter?
\end{itemize}

\section{Future Work}

\begin{itemize}
  \item Perform these tests on the rest of the utilities in busybox and gnu coreutils.
  \item Try testing other programs that don't use stdout and determine if they present any different challenges.
  \item Figure out how to change the VMM parameters for S2E (e.g., give it more memory). Discuss how even building from source had its problems as we could not transfer files from host to guest when we built S2E manually.
\end{itemize}

% conference papers do not normally have an appendix


% use section* for acknowledgement
%\section*{Acknowledgment}


%The authors would like to thank...





% trigger a \newpage just before the given reference
% number - used to balance the columns on the last page
% adjust value as needed - may need to be readjusted if
% the document is modified later
%\IEEEtriggeratref{8}
% The "triggered" command can be changed if desired:
%\IEEEtriggercmd{\enlargethispage{-5in}}

% references section

% can use a bibliography generated by BibTeX as a .bbl file
% BibTeX documentation can be easily obtained at:
% http://www.ctan.org/tex-archive/biblio/bibtex/contrib/doc/
% The IEEEtran BibTeX style support page is at:
% http://www.michaelshell.org/tex/ieeetran/bibtex/
%\bibliographystyle{IEEEtran}
% argument is your BibTeX string definitions and bibliography database(s)
%\bibliography{IEEEabrv,../bib/paper}
%
% <OR> manually copy in the resultant .bbl file
% set second argument of \begin to the number of references
% (used to reserve space for the reference number labels box)
\begin{thebibliography}{1}

%\bibitem{s2e}
%V.~Chipounov, V.~Georgescu, C.~Zamfir, and G.~Candea.
%\emph{Selective symbolic execution}. In Workshop on Hot Topics
%in Dependable Systems, 2009.
%
%\bibitem{klee}
%C.~Cadar, D.~Dunbar, and D.~R. Engler, \emph{KLEE:
%Unassisted and automatic generation of high-coverage tests
%for complex systems programs}. In Symp. on Operating Systems
%Design and Implementation, 2008.

\bibitem{s2e}
CHIPOUNOV, V., GEORGESCU, V., ZAMFIR, C., AND CANDEA, G
Selective symbolic execution. In Workshop on Hot Topics
in Dependable Systems, 2009.

\bibitem{symb-ex-practice}
CADAR, C., GODEFROID, P., KHURSHID, S., PASAREANU, C., SEN, K., TILLMANN, N., AND VISSER, W.
Symbolic execution for software testing in practice � preliminary assessment. In ICSE Impact�11, May 2011.

\bibitem{select}
BOYER, R. S., ELSPAS, B., AND LEVITT, K. N.
SELECT - a formal system for testing and debugging programs by symbolic execution.
SIGPLAN Not., 10:234�245, 1975.

\bibitem{satisfiability}
DE MOURA, L. AND BJ�RNER, N.
Satisfiability modulo theories: introduction and applications.
Commun. ACM, 54:69�77, Sept. 2011.

\bibitem{systems-crash}
CADAR, C. AND ENGLER, D.
Execution generated test cases: How to make systems code crash itself.
In SPIN�05, Aug 2005.

\bibitem{klee}
CADAR, C., DUNBAR, AND ENGLER, D.R.
KLEE: Unassisted and automatic generation of high-coverage tests
for complex systems programs. In Symp. on Operating Systems
Design and Implementation, 2008.

\bibitem{cute}
SEN, K., MARINOV, D., AND AGHA, G. CUTE: A concolic unit testing engine for C. In In 5th joint meeting of the European Software Engineering Conference and ACM Symposium on the Foundations of Software Engineering (ESEC/FSE 2005).

\bibitem{RWset}
BOONSTOPPEL, P., CADAR, C., AND ENGLER, D. RWset: At- tacking path explosion in constraint-based test generation. In Proceedings of Tools and Algorithms for the Construction and Analysis of Systems (TACAS 2008).

\bibitem{vul-signature}
BRUMLEY, D., NEWSOME, J., SONG, D., WANG, H., AND JHA, S. Towards automatic generation of vulnerability-based sig- natures. In Proceedings of the 2006 IEEE Symposium on Security and Privacy (IEEE S\&P 2006).

\bibitem{system-crash}
CADAR, C., AND ENGLER, D. Execution generated test cases: How to make systems code crash itself. In Proceedings of the 12th International SPIN Workshop on Model Checking of Soft- ware (SPIN 2005).

\bibitem{exe}
CADAR, C., GANESH, V., PAWLOWSKI, P., DILL, D., AND ENGLER, D. EXE: Automatically generating inputs of death. In Proceedings of the 13th ACM Conference on Computer and Communications Security (CCS 2006).

\bibitem{bad-input}
COSTA, M., CASTRO, M., ZHOU, L., ZHANG, L., AND PEINADO, M. Bouncer: Securing software by blocking bad input. In Proceedings of the 21th ACM Symposium on Operating Systems Principles (SOSP 2007).

\bibitem{end-to-end}
COSTA, M., CROWCROFT, J., CASTRO, M., ROWSTRON, A., ZHOU, L., ZHANG, L., AND BARHAM, P. Vigilante: end-to-end containment of Internet worms. In Proceedings of the 20th ACM Symposium on Operating Systems Principles (SOSP 2005).

\bibitem{dyn-test}
EMMI, M., MAJUMDAR, R., AND SEN, K. Dynamic test input generation for database applications. In International Symposium on Software Testing and Analysis (ISSTA 2007).

\bibitem{compositional-test}
GODEFROID, P. Compositional dynamic test generation. In Proceedings of the 34th Symposium on Principles of Programming Languages (POPL 2007).

\bibitem{dart}
GODEFROID, P., KLARLUND, N., AND SEN, K. DART: Directed automated random testing. In Proceedings of the Conference on Programming Language Design and Implementation (PLDI 2005).

\bibitem{sage}
GODEFROID, P., LEVIN, M. Y., AND MOLNAR, D. Automated whitebox fuzz testing. In Proceedings of Network and Distributed Systems Security (NDSS 2008).

\bibitem{disassembly}
SCHWARZ, B., DEBRAY, S., AND ANDREWS, G. Disassembly of executable code revisited. In Working Conf. on Reverse Engineering, 2002.

\bibitem{saturn}
DIILLIG, I., DILLIG, T., AND ALKEN, A. Sound, complete and scalable path- sensitive analysis. In Conf. on Programming Language Design and Implementation, 2008.

\bibitem{bddbddb}
LAM, M. S., WHALEY, J., LIVSHITS, V. B., MARTIN, M. C., AVOTS, D., CARBIN, M., AND UNKEL, C. Context-sensitive program analysis as
database queries. In Symp. on Principles of Database Systems, 2005.


\end{thebibliography}

% that's all folks
\end{document}
